% Options for packages loaded elsewhere
\PassOptionsToPackage{unicode}{hyperref}
\PassOptionsToPackage{hyphens}{url}
\PassOptionsToPackage{dvipsnames,svgnames,x11names}{xcolor}
%
\documentclass[
  letterpaper,
  DIV=11,
  numbers=noendperiod]{scrartcl}

\usepackage{amsmath,amssymb}
\usepackage{iftex}
\ifPDFTeX
  \usepackage[T1]{fontenc}
  \usepackage[utf8]{inputenc}
  \usepackage{textcomp} % provide euro and other symbols
\else % if luatex or xetex
  \usepackage{unicode-math}
  \defaultfontfeatures{Scale=MatchLowercase}
  \defaultfontfeatures[\rmfamily]{Ligatures=TeX,Scale=1}
\fi
\usepackage{lmodern}
\ifPDFTeX\else  
    % xetex/luatex font selection
\fi
% Use upquote if available, for straight quotes in verbatim environments
\IfFileExists{upquote.sty}{\usepackage{upquote}}{}
\IfFileExists{microtype.sty}{% use microtype if available
  \usepackage[]{microtype}
  \UseMicrotypeSet[protrusion]{basicmath} % disable protrusion for tt fonts
}{}
\makeatletter
\@ifundefined{KOMAClassName}{% if non-KOMA class
  \IfFileExists{parskip.sty}{%
    \usepackage{parskip}
  }{% else
    \setlength{\parindent}{0pt}
    \setlength{\parskip}{6pt plus 2pt minus 1pt}}
}{% if KOMA class
  \KOMAoptions{parskip=half}}
\makeatother
\usepackage{xcolor}
\setlength{\emergencystretch}{3em} % prevent overfull lines
\setcounter{secnumdepth}{-\maxdimen} % remove section numbering
% Make \paragraph and \subparagraph free-standing
\ifx\paragraph\undefined\else
  \let\oldparagraph\paragraph
  \renewcommand{\paragraph}[1]{\oldparagraph{#1}\mbox{}}
\fi
\ifx\subparagraph\undefined\else
  \let\oldsubparagraph\subparagraph
  \renewcommand{\subparagraph}[1]{\oldsubparagraph{#1}\mbox{}}
\fi

\usepackage{color}
\usepackage{fancyvrb}
\newcommand{\VerbBar}{|}
\newcommand{\VERB}{\Verb[commandchars=\\\{\}]}
\DefineVerbatimEnvironment{Highlighting}{Verbatim}{commandchars=\\\{\}}
% Add ',fontsize=\small' for more characters per line
\usepackage{framed}
\definecolor{shadecolor}{RGB}{241,243,245}
\newenvironment{Shaded}{\begin{snugshade}}{\end{snugshade}}
\newcommand{\AlertTok}[1]{\textcolor[rgb]{0.68,0.00,0.00}{#1}}
\newcommand{\AnnotationTok}[1]{\textcolor[rgb]{0.37,0.37,0.37}{#1}}
\newcommand{\AttributeTok}[1]{\textcolor[rgb]{0.40,0.45,0.13}{#1}}
\newcommand{\BaseNTok}[1]{\textcolor[rgb]{0.68,0.00,0.00}{#1}}
\newcommand{\BuiltInTok}[1]{\textcolor[rgb]{0.00,0.23,0.31}{#1}}
\newcommand{\CharTok}[1]{\textcolor[rgb]{0.13,0.47,0.30}{#1}}
\newcommand{\CommentTok}[1]{\textcolor[rgb]{0.37,0.37,0.37}{#1}}
\newcommand{\CommentVarTok}[1]{\textcolor[rgb]{0.37,0.37,0.37}{\textit{#1}}}
\newcommand{\ConstantTok}[1]{\textcolor[rgb]{0.56,0.35,0.01}{#1}}
\newcommand{\ControlFlowTok}[1]{\textcolor[rgb]{0.00,0.23,0.31}{#1}}
\newcommand{\DataTypeTok}[1]{\textcolor[rgb]{0.68,0.00,0.00}{#1}}
\newcommand{\DecValTok}[1]{\textcolor[rgb]{0.68,0.00,0.00}{#1}}
\newcommand{\DocumentationTok}[1]{\textcolor[rgb]{0.37,0.37,0.37}{\textit{#1}}}
\newcommand{\ErrorTok}[1]{\textcolor[rgb]{0.68,0.00,0.00}{#1}}
\newcommand{\ExtensionTok}[1]{\textcolor[rgb]{0.00,0.23,0.31}{#1}}
\newcommand{\FloatTok}[1]{\textcolor[rgb]{0.68,0.00,0.00}{#1}}
\newcommand{\FunctionTok}[1]{\textcolor[rgb]{0.28,0.35,0.67}{#1}}
\newcommand{\ImportTok}[1]{\textcolor[rgb]{0.00,0.46,0.62}{#1}}
\newcommand{\InformationTok}[1]{\textcolor[rgb]{0.37,0.37,0.37}{#1}}
\newcommand{\KeywordTok}[1]{\textcolor[rgb]{0.00,0.23,0.31}{#1}}
\newcommand{\NormalTok}[1]{\textcolor[rgb]{0.00,0.23,0.31}{#1}}
\newcommand{\OperatorTok}[1]{\textcolor[rgb]{0.37,0.37,0.37}{#1}}
\newcommand{\OtherTok}[1]{\textcolor[rgb]{0.00,0.23,0.31}{#1}}
\newcommand{\PreprocessorTok}[1]{\textcolor[rgb]{0.68,0.00,0.00}{#1}}
\newcommand{\RegionMarkerTok}[1]{\textcolor[rgb]{0.00,0.23,0.31}{#1}}
\newcommand{\SpecialCharTok}[1]{\textcolor[rgb]{0.37,0.37,0.37}{#1}}
\newcommand{\SpecialStringTok}[1]{\textcolor[rgb]{0.13,0.47,0.30}{#1}}
\newcommand{\StringTok}[1]{\textcolor[rgb]{0.13,0.47,0.30}{#1}}
\newcommand{\VariableTok}[1]{\textcolor[rgb]{0.07,0.07,0.07}{#1}}
\newcommand{\VerbatimStringTok}[1]{\textcolor[rgb]{0.13,0.47,0.30}{#1}}
\newcommand{\WarningTok}[1]{\textcolor[rgb]{0.37,0.37,0.37}{\textit{#1}}}

\providecommand{\tightlist}{%
  \setlength{\itemsep}{0pt}\setlength{\parskip}{0pt}}\usepackage{longtable,booktabs,array}
\usepackage{calc} % for calculating minipage widths
% Correct order of tables after \paragraph or \subparagraph
\usepackage{etoolbox}
\makeatletter
\patchcmd\longtable{\par}{\if@noskipsec\mbox{}\fi\par}{}{}
\makeatother
% Allow footnotes in longtable head/foot
\IfFileExists{footnotehyper.sty}{\usepackage{footnotehyper}}{\usepackage{footnote}}
\makesavenoteenv{longtable}
\usepackage{graphicx}
\makeatletter
\def\maxwidth{\ifdim\Gin@nat@width>\linewidth\linewidth\else\Gin@nat@width\fi}
\def\maxheight{\ifdim\Gin@nat@height>\textheight\textheight\else\Gin@nat@height\fi}
\makeatother
% Scale images if necessary, so that they will not overflow the page
% margins by default, and it is still possible to overwrite the defaults
% using explicit options in \includegraphics[width, height, ...]{}
\setkeys{Gin}{width=\maxwidth,height=\maxheight,keepaspectratio}
% Set default figure placement to htbp
\makeatletter
\def\fps@figure{htbp}
\makeatother

\KOMAoption{captions}{tableheading}
\makeatletter
\makeatother
\makeatletter
\makeatother
\makeatletter
\@ifpackageloaded{caption}{}{\usepackage{caption}}
\AtBeginDocument{%
\ifdefined\contentsname
  \renewcommand*\contentsname{Table of contents}
\else
  \newcommand\contentsname{Table of contents}
\fi
\ifdefined\listfigurename
  \renewcommand*\listfigurename{List of Figures}
\else
  \newcommand\listfigurename{List of Figures}
\fi
\ifdefined\listtablename
  \renewcommand*\listtablename{List of Tables}
\else
  \newcommand\listtablename{List of Tables}
\fi
\ifdefined\figurename
  \renewcommand*\figurename{Figure}
\else
  \newcommand\figurename{Figure}
\fi
\ifdefined\tablename
  \renewcommand*\tablename{Table}
\else
  \newcommand\tablename{Table}
\fi
}
\@ifpackageloaded{float}{}{\usepackage{float}}
\floatstyle{ruled}
\@ifundefined{c@chapter}{\newfloat{codelisting}{h}{lop}}{\newfloat{codelisting}{h}{lop}[chapter]}
\floatname{codelisting}{Listing}
\newcommand*\listoflistings{\listof{codelisting}{List of Listings}}
\makeatother
\makeatletter
\@ifpackageloaded{caption}{}{\usepackage{caption}}
\@ifpackageloaded{subcaption}{}{\usepackage{subcaption}}
\makeatother
\makeatletter
\@ifpackageloaded{tcolorbox}{}{\usepackage[skins,breakable]{tcolorbox}}
\makeatother
\makeatletter
\@ifundefined{shadecolor}{\definecolor{shadecolor}{rgb}{.97, .97, .97}}
\makeatother
\makeatletter
\makeatother
\ifLuaTeX
  \usepackage{selnolig}  % disable illegal ligatures
\fi
\IfFileExists{bookmark.sty}{\usepackage{bookmark}}{\usepackage{hyperref}}
\IfFileExists{xurl.sty}{\usepackage{xurl}}{} % add URL line breaks if available
\urlstyle{same} % disable monospaced font for URLs
\hypersetup{
  pdftitle={json-ld},
  colorlinks=true,
  linkcolor={blue},
  filecolor={Maroon},
  citecolor={Blue},
  urlcolor={Blue},
  pdfcreator={LaTeX via pandoc}}

\title{json-ld}
\usepackage{etoolbox}
\makeatletter
\providecommand{\subtitle}[1]{% add subtitle to \maketitle
  \apptocmd{\@title}{\par {\large #1 \par}}{}{}
}
\makeatother
\subtitle{metadata format for linked data}
\author{}
\date{}

\begin{document}
\maketitle
\numberwithin{algorithm}{chapter}
\algrenewcommand{\algorithmiccomment}[1]{\hskip3em$\rightarrow$ #1}

\ifdefined\Shaded\renewenvironment{Shaded}{\begin{tcolorbox}[frame hidden, boxrule=0pt, borderline west={3pt}{0pt}{shadecolor}, interior hidden, breakable, sharp corners, enhanced]}{\end{tcolorbox}}\fi

I've been excited about JSON-LD, when it came out as it offered a much
cleaner way to do SEO. However It dawned on me later that it is also a
geat format to worked with linked data.

\begin{enumerate}
\def\labelenumi{\arabic{enumi}.}
\tightlist
\item
  it uses a special \texttt{@id} property to assign a unique URL to each
  resource in a JSON-LD document, giving every data item its own URL.
\item
  it links data items together through the values of properties. For
  example, if you're describing a person, their ``colleague'' property
  can have another person's URL as its value, creating a web of
  interconnected data items.
\item
  JSON-LD uses the \texttt{@context} property to map the terms used in
  the document to URLs, conforming data items to common global,
  organisational and departmental schemas.
\end{enumerate}

One of the best things about JSON-LD is how easy it is to work with.
Developers already familiar with JSON syntax will love using it. And get
this: JSON-LD is so popular that it's now embedded in almost half of all
web pages. It's baffling why all organisations aren't using JSON-LD more
widely to share data between their applications!

So, whether you're publishing Data Products, creating RESTful
applications or improving your website's SEO, JSON-LD is the way to go!
Give it a try and let me know your thoughts in the comments below.

\hypertarget{annotated-cell-1}{%
\label{annotated-cell-1}}%
\begin{Shaded}
\begin{Highlighting}[]

\NormalTok{\{}
  \StringTok{"@context"}\NormalTok{: }\StringTok{"https://json{-}ld.org/contexts/person.jsonld"}\NormalTok{,}\hspace*{\fill}\NormalTok{\circled{1}}
  \StringTok{"@type"}\NormalTok{: }\StringTok{"https://schema.org/Person"}\NormalTok{,}\hspace*{\fill}\NormalTok{\circled{2}}
  \StringTok{"@id"}\NormalTok{: }\StringTok{"http://dbpedia.org/resource/John\_Lennon"}\NormalTok{,}\hspace*{\fill}\NormalTok{\circled{3}}
  \StringTok{"https://schema.org/name"}\NormalTok{: }\StringTok{"John Lennon"}\NormalTok{,}
  \StringTok{"born"}\NormalTok{: }\StringTok{"1940{-}10{-}09"}\NormalTok{,}
  \StringTok{"spouse"}\NormalTok{: }\StringTok{"http://dbpedia.org/resource/Cynthia\_Lennon"}\hspace*{\fill}\NormalTok{\circled{4}}
\NormalTok{\}}
\end{Highlighting}
\end{Shaded}

\begin{description}
\tightlist
\item[\circled{1}]
the \texttt{@context} refrences your model
\item[\circled{2}]
the \texttt{@type} is the type in your model
\item[\circled{3}]
the \texttt{@id} is the yrl for this item
\item[\circled{4}]
the url is how to reference external data say dbpedia.
\end{description}

\hypertarget{python}{%
\section{python}\label{python}}

\begin{Shaded}
\begin{Highlighting}[]
\ImportTok{from}\NormalTok{ pyld }\ImportTok{import}\NormalTok{ jsonld}
\ImportTok{import}\NormalTok{ json}

\NormalTok{doc }\OperatorTok{=}\NormalTok{ \{}
    \StringTok{"http://schema.org/name"}\NormalTok{: }\StringTok{"Manu Sporny"}\NormalTok{,}
    \StringTok{"http://schema.org/url"}\NormalTok{: \{}\StringTok{"@id"}\NormalTok{: }\StringTok{"http://manu.sporny.org/"}\NormalTok{\},}
    \StringTok{"http://schema.org/image"}\NormalTok{: \{}\StringTok{"@id"}\NormalTok{: }\StringTok{"http://manu.sporny.org/images/manu.png"}\NormalTok{\}}
\NormalTok{\}}

\NormalTok{context }\OperatorTok{=}\NormalTok{ \{}
    \StringTok{"name"}\NormalTok{: }\StringTok{"http://schema.org/name"}\NormalTok{,}
    \StringTok{"homepage"}\NormalTok{: \{}\StringTok{"@id"}\NormalTok{: }\StringTok{"http://schema.org/url"}\NormalTok{, }\StringTok{"@type"}\NormalTok{: }\StringTok{"@id"}\NormalTok{\},}
    \StringTok{"image"}\NormalTok{: \{}\StringTok{"@id"}\NormalTok{: }\StringTok{"http://schema.org/image"}\NormalTok{, }\StringTok{"@type"}\NormalTok{: }\StringTok{"@id"}\NormalTok{\}}
\NormalTok{\}}

\CommentTok{\# compact a document according to a particular context}
\CommentTok{\# see: https://json{-}ld.org/spec/latest/json{-}ld/\#compacted{-}document{-}form}
\NormalTok{compacted }\OperatorTok{=}\NormalTok{ jsonld.compact(doc, context)}

\BuiltInTok{print}\NormalTok{(json.dumps(compacted, indent}\OperatorTok{=}\DecValTok{2}\NormalTok{))}
\end{Highlighting}
\end{Shaded}

\begin{verbatim}
{
  "@context": {
    "name": "http://schema.org/name",
    "homepage": {
      "@id": "http://schema.org/url",
      "@type": "@id"
    },
    "image": {
      "@id": "http://schema.org/image",
      "@type": "@id"
    }
  },
  "image": "http://manu.sporny.org/images/manu.png",
  "name": "Manu Sporny",
  "homepage": "http://manu.sporny.org/"
}
\end{verbatim}

\hypertarget{output}{%
\section{Output:}\label{output}}

\hypertarget{section}{%
\section{\{}\label{section}}

\hypertarget{context}{%
\section{``@context'': \{\ldots\},}\label{context}}

\hypertarget{image-httpmanu.sporny.orgimagesmanu.png}{%
\section{``image'':
``http://manu.sporny.org/images/manu.png'',}\label{image-httpmanu.sporny.orgimagesmanu.png}}

\hypertarget{homepage-httpmanu.sporny.org}{%
\section{``homepage'':
``http://manu.sporny.org/'',}\label{homepage-httpmanu.sporny.org}}

\hypertarget{name-manu-sporny}{%
\section{``name'': ``Manu Sporny''}\label{name-manu-sporny}}

\hypertarget{section-1}{%
\section{\}}\label{section-1}}

\begin{verbatim}
# compact using URLs
jsonld.compact('http://example.org/doc', 'http://example.org/context')

# expand a document, removing its context
# see: https://json-ld.org/spec/latest/json-ld/#expanded-document-form
expanded = jsonld.expand(compacted)

print(json.dumps(expanded, indent=2))
# Output:
# [{
#   "http://schema.org/image": [{"@id": "http://manu.sporny.org/images/manu.png"}],
#   "http://schema.org/name": [{"@value": "Manu Sporny"}],
#   "http://schema.org/url": [{"@id": "http://manu.sporny.org/"}]
# }]

# expand using URLs
jsonld.expand('http://example.org/doc')

# flatten a document
# see: https://json-ld.org/spec/latest/json-ld/#flattened-document-form
flattened = jsonld.flatten(doc)
# all deep-level trees flattened to the top-level

# frame a document
# see: https://json-ld.org/spec/latest/json-ld-framing/#introduction
framed = jsonld.frame(doc, frame)
# document transformed into a particular tree structure per the given frame

# normalize a document using the RDF Dataset Normalization Algorithm
# (URDNA2015), see: https://json-ld.github.io/normalization/spec/
normalized = jsonld.normalize(
    doc, {'algorithm': 'URDNA2015', 'format': 'application/n-quads'})
# normalized is a string that is a canonical representation of the document
# that can be used for hashing, comparison, etc.
\end{verbatim}

\hypertarget{resources}{%
\section{resources}\label{resources}}

\begin{enumerate}
\def\labelenumi{\arabic{enumi}.}
\tightlist
\item
  \href{https://json-ld.org/}{json-ld}
\item
  \href{https://schema.org/}{schema.org}
\item
  \href{http://www.productontology.org/}{ProductOntology}
\item
  \href{http://www.heppnetz.de/projects/goodrelations/}{GoodRelations}
\item
  \href{https://www.wikidata.org/wiki/Wikidata:Main_Page}{Wikidata}
\item
  https://schemantra.com/
\item
  https://github.com/science-periodicals/jsonld-vis
\item
  https://github.com/shamilnabiyev/schema-visualizer
\item
  https://cloud.google.com/natural-language/
\end{enumerate}



\end{document}
